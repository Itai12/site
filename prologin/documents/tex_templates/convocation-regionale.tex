\documentclass[a4paper,11pt]{article}
\usepackage{prologin-letter}

\def\objet{Convocation aux épreuves régionales de Prologin #YEAR#}

\begin{document}

% \convocation definition HAS to be AFTER \begin{document}
% for babel transformations to work

\newcommand\convocation[9]{
\def\destinataire{#1 #2\\
#3\\
#4 #5}
\header

Bonjour #1,

Nous avons le plaisir de vous informer que vous êtes sélectionné
pour la phase des épreuves régionales du concours Prologin 2013.
Comme vous nous l'avez demandé, votre épreuve aura lieu à:

{\par\smallskip\noindent\centering
\begin{minipage}{0.5\textwidth}
\textbf{#7}\\
\textbf{#8}\\
\textbf{#9}
\end{minipage}
\par\smallskip}

L'équipe de Prologin vous donne donc rendez-vous le #6
à partir de 8~h~45 (début des épreuves à 9~heures). Nous vous demandons
de bien vouloir vous munir d'\textbf{une pièce d'identité et de cette convocation}.

Après un petit déjeuner offert par Prologin, la journée commencera par une
épreuve écrite d'algorithmique d'une durée de 3~heures et un entretien de
20~minutes. Le repas du midi sera offert par l'association. Les épreuves
sur machine (3~h~30) se dérouleront l'après-midi. Vous serez libérés vers
17~h~30 environ.

Pour vous préparer à l'épreuve, nous vous invitons à utiliser le site
d'entraînement de Prologin: \url{http://www.prologin.org}.

Nous restons à votre disposition pour toutes questions, n'hésitez pas
à nous contacter.

\signature
\newpage}


#BODY#
\end{document}
