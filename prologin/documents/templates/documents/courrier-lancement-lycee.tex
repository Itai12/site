\documentclass[9pt]{article}
\usepackage[utf8]{inputenc}
\usepackage[french]{babel}
\usepackage[T1]{fontenc}
\usepackage{fancyhdr}
\usepackage{hyperref}
\usepackage{helvet}
\usepackage{geometry}
\usepackage{graphicx}
\usepackage{multicol}
\setlength\columnsep{7pt}
\renewcommand{\familydefault}{\sfdefault}

\geometry{margin=2.5cm}

\usepackage[table,dvipsnames]{xcolor}

\newcommand*{\vcenteredhbox}[1]{\begingroup
\setbox0=\hbox{#1}\parbox{\wd0}{\box0}\endgroup}

\pagestyle{fancy}
\fancyhf{}
\renewcommand\headrulewidth{0pt}
\cfoot{\begin{footnotesize}
 Association Prologin\\
 14-16 rue Voltaire — 94270 Le Kremlin-Bicêtre — France\\
 Tél. : 01 44 08 01 90 — E-mail : info@prologin.org — http://prologin.org\\
\end{footnotesize}}

\begin{document}
 \begin{flushright}
 Paris, le \today
 \end{flushright}

 \hspace{-1cm}
 \includegraphics[width=6.5cm]{logo{{year}}.pdf}
 \vspace{1cm}

 \noindent
 \textbf{\underline{Objet :} Lancement de l'édition {{ year }} de Prologin, le concours national d’informatique}

 \vspace{1cm}
 \noindent
 \textbf{À l’attention des professeurs de mathématiques et d’informatique de votre établissement}\\

 Madame, monsieur,\\

 \noindent
 J’ai le plaisir de vous annoncer le lancement de l'édition {{ year }} du
 concours national d’informatique \textbf{Prologin}. Ouvert à tous les jeunes de 20
 ans et moins, ce concours entièrement gratuit est l’occasion pour vos
 élèves de s’initier à la programmation, démontrer leurs talents et de
 rencontrer d’autres passionnés d’informatique.\\

 \noindent
 Chaque année, les finalistes sont récompensés par du matériel informatique
 (ordinateurs portables, liseuses, consoles), des livres et bien d’autres
 lots. Ils ont également l’opportunité de rencontrer nos sponsors, le
 concours étant reconnu par la presse et l’industrie informatique.\\

 \noindent
 Pour participer, il suffit de remplir le questionnaire sur
 \url{http://prologin.org} \textbf{avant le {{ qualif.date_end|date:"d F Y" }}}.\\

 \noindent
 Nous faisons donc appel à vous pour informer vos élèves de l’existence de
 ce concours, notamment si vous enseignez la spécialité Informatique et
 Sciences du Numérique en Terminale S. Nous espérons que nos ressources leur
 permettront de progresser dans les domaines de la programmation et de
 l’algorithmique.\\

 \noindent
 Faire participer votre classe entière peut être une excellente méthode
 d’apprentissage pour vos élèves : les exercices de sélection, ordonnés par
 difficulté croissante, sont corrigés automatiquement par notre site web, ce
 qui leur permet de constater directement leurs erreurs et de les corriger.
 Vous pouvez bien entendu les assister s’ils rencontrent des difficultés
 pendant les exercices, à condition qu’ils écrivent eux-mêmes les programmes
 de l’épreuve de sélection.\\

 \noindent
 Vous trouverez ci-joint l’affiche officielle du concours. Pour toutes
 questions, n’hésitez pas à nous contacter à l’adresse
 \url{info@prologin.org}.\\

 Je vous prie d’agréer, Madame, Monsieur, l’expression de mes salutations
 les plus distinguées.\\

 \begin{flushright}
 Pour l’équipe Prologin
 \end{flushright}

 \begin{flushright}
 {{ pres.title_name }} {{ pres.user.first_name|title }} {{ pres.user.last_name|title }}
 \vcenteredhbox{\includegraphics[width=2.5cm]{signature_{{ year }}.png}}
 \end{flushright}

 %\vspace{5mm}

 \begin{multicols}{ {{ sponsors|length }} }
 
 \noindent
 \includegraphics[width=.9\columnwidth]{{ "{" }}{{ sponsor.logo.path }}}
 
 \end{multicols}
\end{document}
