
\documentclass[a4paper,11pt]{article}
\usepackage{prologin-letter}
\usepackage[french]{babel}
\usepackage[pdftex,
            pdfauthor={Prologin},
            pdftitle={ {{ pdf_title }} }]{hyperref}

\begin{document}

\def\objet{Convocation à la finale Prologin {{ year }} }




\def\destinataire{%
{{ user.get_full_name|title|escapetex|nonempty }}\\
{{ user.address|escapetex|nonempty }}\\
{{ user.postal_code|escapetex|nonempty }} {{ user.city|escapetex|nonempty }}%
}
\header

Bonjour {{ user.first_name|title|escapetex }},

Nous avons le plaisir de vous informer que vous êtes sélectionné
pour la finale du concours Prologin 2013, qui se déroulera à:

{\par\smallskip\noindent\centering
\begin{minipage}{0.5\textwidth}
\textbf{%
{{ center.name|escapetex|nonempty }}%
}\\
\textbf{%
{{ center.address|escapetex|nonempty }}%
}\\
\textbf{%
{{ center.postal_code|escapetex|nonempty }} {{ center.city|escapetex|nonempty }}%
}
\end{minipage}
\par\smallskip}

L'équipe de Prologin vous donne donc rendez-vous le {{ event.date_begin|date:"l d F Y"|escapetex }}
à 9~h~00. Le concours, d'une durée de 36~heures, prendra fin avec le banquet du
lundi midi. Vous serez donc libre à partir de 15~h~00. Notez que la
programmation se fera obligatoirement en C, C++, Java, Python, PHP, C\#, ou
en Caml. Nous vous demandons de bien vouloir vous munir d'\textbf{une pièce d'identité
et de cette convocation, ainsi que de tous les documents ci-joints dûment
remplis}. Attention, sans ces documents, nous ne pourrons pas vous
accepter!

Si vous le souhaitez, il vous est possible de passer la nuit du vendredi dans les locaux. Les
organisateurs Prologin assureront l'accueil à partir de 17~h~00 et
proposeront entre autres des formations GNU/Linux pendant la soirée.

En attendant, je vous invite à découvrir sur le site \texttt{www.prologin.org}
les sujets des finales des années précédentes.
Nous nous occuperons de vous durant ces trois jours pour tout ce qui concerne la
nourriture et le logement. Voici quelques recommandations quant aux affaires
dont vous aurez besoin:
\begin{itemize}
\item deux tenues complètes de rechange;
\item une tenue que vous n'avez pas peur de salir un peu;
\item une grande serviette de bain, ainsi qu'une brosse à dents;
\item un sac de couchage (nous fournissons le matelas);
\item une bonne dose de courage et de bonne humeur!
\end{itemize}
\vspace{0.5cm}
Nous restons à votre disposition pour toutes questions, n'hésitez pas
à nous contacter.

\signature
\newpage




\end{document}

