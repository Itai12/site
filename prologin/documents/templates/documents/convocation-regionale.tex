
\documentclass[a4paper,11pt]{article}
\usepackage{prologin-letter}

\begin{document}

\def\objet{Convocation aux épreuves régionales de Prologin {{ year|escapetex }} }




\def\destinataire{%
{{ user.get_full_name|title|escapetex|nonempty }}\\
{{ user.address|escapetex|nonempty }}\\
{{ user.postal_code|escapetex|nonempty }} {{ user.city|escapetex|nonempty }}%
}
\header

Bonjour {{ user.first_name|title|escapetex }},

Nous avons le plaisir de vous informer que vous êtes sélectionné
pour la phase des épreuves régionales du concours Prologin {{ year|escapetex|nonempty }}.
Comme vous nous l'avez demandé, votre épreuve aura lieu à:

{\par\smallskip\noindent\centering
\begin{minipage}{0.5\textwidth}
\begin{center}
\textbf{%
{{ center.name|escapetex|nonempty }}%
}\\
\textbf{%
{{ center.address|escapetex|nonempty }}%
}\\
\textbf{%
{{ center.postal_code|escapetex|nonempty }} {{ center.city|escapetex|nonempty }}%
}
\end{center}
\end{minipage}
\par\smallskip}

L'équipe de Prologin vous donne donc rendez-vous le {{ event.date_begin|date:"l d F Y"|escapetex }}
à partir de 8~h~45 (début des épreuves à 9~heures). Nous vous demandons
de bien vouloir vous munir d'\textbf{une pièce d'identité et de cette convocation}.

Après un petit déjeuner offert par Prologin, la journée commencera par une
épreuve écrite d'algorithmique d'une durée de 3~heures et un entretien de
20~minutes. Le repas du midi sera offert par l'association. Les épreuves
sur machine (3~h~30) se dérouleront l'après-midi. Vous serez libérés vers
17~h~30 environ.

Pour vous préparer à l'épreuve, nous vous invitons à utiliser le site
d'entraînement de Prologin: \url{http://www.prologin.org}.

Nous restons à votre disposition pour toutes questions, n'hésitez pas
à nous contacter.

\signature
\newpage




\end{document}

